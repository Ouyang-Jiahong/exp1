\usepackage{amsmath}                                    % 排版数学公式

% 使equation计数器也依赖于section计数器
\numberwithin{equation}{section}

% 使table计数器也依赖于section计数器
\numberwithin{table}{section}

% 使figure计数器也依赖于section计数器
\numberwithin{figure}{section}

\usepackage{latexsym}
\usepackage{amsfonts}                                   %数学符号字体库宏包套件,它包含有:amsfonts、amssymb、eufrak 和 eucal 四个宏包。
\usepackage{amssymb}                                    % 定义AMS的数学符号命令
\usepackage{mathrsfs}                                   % 数学RSFS书写字体
\usepackage{bm}                                         % 数学黑体
\usepackage{graphicx}                                   % 支持插图,图形宏包graphics的扩展宏包
\usepackage{color,xcolor}                               % 支持彩色
\usepackage{amscd}
\usepackage[linesnumbered,ruled,vlined]{algorithm2e}
\usepackage{diagbox}
\usepackage{minted}
\usepackage{titlesec}                                   %设置章节格式
\usepackage{tcolorbox}

% 设置 paragraph 为只有一级编号(从1开始)
\renewcommand{\theparagraph}{\arabic{paragraph}} % 只显示一级编号
\makeatletter
\@addtoreset{paragraph}{section} % 每节开始时重置 paragraph 计数器
\makeatother

%%文本框设置
\newcommand{\tbox}[1]{
  \begin{center}
    \begin{tcolorbox}[colback=gray!10,%gray background
        colframe=black,% black frame colour
        width=14cm,% Use 8cm total width,
        arc=1mm, auto outer arc,
        boxrule=0.5pt,
      ]
      {#1}
    \end{tcolorbox}
  \end{center}
}

% 设置 paragraph 为 block 风格,自动换行
\titleformat{\paragraph}[block]
  {\normalfont\normalsize\bfseries}{\theparagraph.}{0.5em}{}

% 设置标题后换行(after-sep 设为 \newline)
\titlespacing*{\paragraph}{0pt}{3.25ex plus 1ex minus .2ex}{0pt}
  
\usepackage{enumerate}                                 	%更改enumerate环境格式
\usepackage{hyperref}

%改变超链接颜色
\hypersetup{
    colorlinks=true,
    linkcolor=blue,
    filecolor=blue,      
    urlcolor=blue,
    citecolor=cyan,
}

\usepackage{subcaption}
\usepackage{minipage-marginpar}
\usepackage{float}%提供float浮动环境
\usepackage{booktabs}%提供命令\toprule、\midrule、\bottomrule
\usepackage{listings}
\usepackage{xcolor}
\usepackage{tabularx}
\usepackage{multirow}
\usepackage[perpage]{footmisc}

% 自定义命令,用于角标引用文献和交叉引用
\newcommand{\scite}[1]{\textsuperscript{\cite{#1}}}
\newcommand{\sref}[1]{\textsuperscript{\ref{#1}}}
\newcommand{\bs}[1]{\boldsymbol{#1}}
\newcommand{\romannumber}[1]{\uppercase\expandafter{\romannumeral#1}}
\newcommand{\alert}[1]{\textcolor{red}{#1}}

% 设置标题层级和目录层级
\setcounter{tocdepth}{3}
\setcounter{secnumdepth}{4}