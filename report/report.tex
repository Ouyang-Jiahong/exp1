\documentclass[openany,12pt,UTF8]{ctexart}
\usepackage[a4paper,margin=2.5cm]{geometry}
\usepackage{amsmath}                                    % 排版数学公式

% 使equation计数器也依赖于section计数器
\numberwithin{equation}{section}

% 使table计数器也依赖于section计数器
\numberwithin{table}{section}

% 使figure计数器也依赖于section计数器
\numberwithin{figure}{section}

\usepackage{latexsym}
\usepackage{amsfonts}                                   %数学符号字体库宏包套件,它包含有:amsfonts、amssymb、eufrak 和 eucal 四个宏包。
\usepackage{amssymb}                                    % 定义AMS的数学符号命令
\usepackage{mathrsfs}                                   % 数学RSFS书写字体
\usepackage{bm}                                         % 数学黑体
\usepackage{graphicx}                                   % 支持插图,图形宏包graphics的扩展宏包
\usepackage{color,xcolor}                               % 支持彩色
\usepackage{amscd}
\usepackage[linesnumbered,ruled,vlined]{algorithm2e}
\usepackage{diagbox}
\usepackage{minted}
\usepackage{titlesec}                                   %设置章节格式
\usepackage{tcolorbox}

% 设置 paragraph 为只有一级编号(从1开始)
\renewcommand{\theparagraph}{\arabic{paragraph}} % 只显示一级编号
\makeatletter
\@addtoreset{paragraph}{section} % 每节开始时重置 paragraph 计数器
\makeatother

%%文本框设置
\newcommand{\tbox}[1]{
  \begin{center}
    \begin{tcolorbox}[colback=gray!10,%gray background
        colframe=black,% black frame colour
        width=14cm,% Use 8cm total width,
        arc=1mm, auto outer arc,
        boxrule=0.5pt,
      ]
      {#1}
    \end{tcolorbox}
  \end{center}
}

% 设置 paragraph 为 block 风格,自动换行
\titleformat{\paragraph}[block]
  {\normalfont\normalsize\bfseries}{\theparagraph.}{0.5em}{}

% 设置标题后换行(after-sep 设为 \newline)
\titlespacing*{\paragraph}{0pt}{3.25ex plus 1ex minus .2ex}{0pt}
  
\usepackage{enumerate}                                 	%更改enumerate环境格式
\usepackage{hyperref}

%改变超链接颜色
\hypersetup{
    colorlinks=true,
    linkcolor=blue,
    filecolor=blue,      
    urlcolor=blue,
    citecolor=cyan,
}

\usepackage{subcaption}
\usepackage{minipage-marginpar}
\usepackage{float}%提供float浮动环境
\usepackage{booktabs}%提供命令\toprule、\midrule、\bottomrule
\usepackage{listings}
\usepackage{xcolor}
\usepackage{tabularx}
\usepackage{multirow}
\usepackage[perpage]{footmisc}

% 自定义命令,用于角标引用文献和交叉引用
\newcommand{\scite}[1]{\textsuperscript{\cite{#1}}}
\newcommand{\sref}[1]{\textsuperscript{\ref{#1}}}
\newcommand{\bs}[1]{\boldsymbol{#1}}
\newcommand{\romannumber}[1]{\uppercase\expandafter{\romannumeral#1}}
\newcommand{\alert}[1]{\textcolor{red}{#1}}
%%对一些autoref的中文引用名作修改
\def\equationautorefname{式}
\def\footnoteautorefname{脚注}
\def\itemautorefname{项}
\def\figureautorefname{图}
\def\tableautorefname{表}
\def\partautorefname{篇}
\def\appendixautorefname{附录}
\def\chapterautorefname{章}
\def\sectionautorefname{节}
\def\subsectionautorefname{小小节}
\def\subsubsectionautorefname{subsubsection}
\def\paragraphautorefname{段落}
\def\subparagraphautorefname{子段落}
\def\FancyVerbLineautorefname{行}
\def\theoremautorefname{定理}

% 设置标题层级和目录层级
\setcounter{tocdepth}{3}
\setcounter{secnumdepth}{4}
\usepackage{rotating}
\title{实验一报告}
\author{欧阳嘉鸿}
\date{\today}
\begin{document}
\maketitle
\newpage
\tableofcontents
\newpage

\section{作业说明}
\subsection{背景}
\subsubsection{陀螺仪的测量模型}
\begin{equation}
    \label{equation:陀螺仪的测量模型}
    \omega_{G} = \underbrace{E_{\omega 0}}_{\text{Bias}} + \underbrace{E_{\omega 1}}_{\text{Factor}} \cdot (\omega + \text{Vel}_{ver}) + \varepsilon_{G}
\end{equation}

\subsubsection{输入数据}
\paragraph{角速度输入}\
\begin{equation}
    \omega = [60\quad 40\quad 50\quad 80\quad 33\quad 24\quad 17]^{\text{T}} \quad (\circ/\text{s})
\end{equation}

\paragraph{测量噪声}\
\begin{equation}
    \varepsilon_{G} \sim N(0,\ 1 \times 10^{-8})
\end{equation}
即均值为0、方差为 $1 \times 10^{-8}$ 的高斯白噪声

\subsubsection{纬度与垂直方向角速度}
\paragraph{地理纬度(PhiN)}\
\begin{equation}
    \text{PhiN} = 28.209167\ ^\circ
\end{equation}
\paragraph{垂直方向角速度(考虑地球自转)}\
\begin{equation}
    \text{Vel}_{ver} = 7.292 \times 10^{-5} \times \text{r2d} \times \sin(\text{PhiN} \times \text{d2r})
\end{equation}
其中:
\begin{itemize}
    \item \texttt{r2d} 表示将弧度转换为角度(即 $180/\pi$)
    \item \texttt{d2r} 表示将角度转换为弧度(即 $\pi/180$)
\end{itemize}

\paragraph{采样步长}\

0.01秒

\subsection{数据格式说明}
\subsubsection{GyroMeasData\_X.mat}
在\texttt{GyroMeasData\_X.mat}中,存储了500×7的double数据,每一组数据1×7代表了一组\(\omega\)观测数据。X代表了实验的组数,从0-100。我们要利用X组实验中的500次观测数据进行Bias和Factor的估计,估计式如\autoref{equation:陀螺仪的测量模型}所示。
\subsubsection{GyroBias.mat}
在\texttt{GyroBias.mat}中,存储了1×100的double数据,每一组数据代表了一个实验的Bias真值。例如GyroBias[1]代表了X=1时的Bias真值,也就是对应GyroMeasData\_1.mat的实验的真值。
在利用\texttt{GyroMeasData\_X.mat}的数据求解处\texttt{estimate\_{bias}}之后,要与真值进行对比。
\subsubsection{GyroFactor.mat}
在\texttt{GyroFactor.mat}中,存储了1×100的double数据,每一组数据代表了一个实验的Factor真值。例如GyroFactor[1]代表了X=1时的Factor真值,也就是对应GyroMeasData\_1.mat的实验的真值。
在利用\texttt{GyroMeasData\_X.mat}的数据求解处\texttt{estimate\_{factor}}之后,要与真值进行对比。
\subsection{任务}
\begin{enumerate}
    \item 请使用批量最小二乘法(Batch LS)和递推最小二乘法(RLS)来估计陀螺仪的偏差(bias)和标度因子(scale factor),并比较这两种估计方法的结果。
    \item 当使用批量最小二乘法(Batch LS)时,请估计测量噪声的方差。
    \item 当使用递推最小二乘法(RLS)时,请通过仿真回答以下问题:
          \begin{itemize}
              \item 当初始参数设置为批量最小二乘(BLS)解时,与初始参数随机设定时,估计误差曲线会如何变化?
              \item 参数估计误差曲线受初始条件的影响是怎样的?
          \end{itemize}
\end{enumerate}

\end{document}